\documentclass{article}

\usepackage{amsmath}
\usepackage{graphicx}
\usepackage{pifont}
\usepackage{amssymb}
\usepackage{listings}
\usepackage[margin=1.0in]{geometry}
\usepackage[table]{xcolor}

\begin{document}
\noindent Rick Sullivan         \\
          Professor Tran 		\\
          Algorithms-COEN 179 	\\
          30 April 2013 		\\
			                    \\
\centerline{Homework \# 2}      \\

\begin{enumerate}
    \item
        \begin{enumerate} 
            \item
                \textbf{Quick sort:}        \\
                \begin{tabular}{ c | c | c }
                    \hline
                    $1_a$ & $1_b$ & $1_c$   \\
                    \hline
                \end{tabular}
                Begin with different elements of same sort value.               \\ \\

                \begin{tabular}{ c | c | c }
                    \hline
                    $1_a$ & $1_b$ & $1_c$  \\ 
                    \hline
                \end{tabular}
                Do nothing as you move through the array (each element = pivot). \\ \\

                \begin{tabular}{ c | c | c }
                    \hline
                    $1_c$ & $1_b$ & $1_a$  \\ 
                    \hline
                \end{tabular}
                At the end, swap i and the pivot.   \\ \\

                Now ordering is not preserved. \\

            \item
                \textbf{Heap sort:} \\
                \begin{tabular}{ c | c | c }
                    \hline
                    $1_a$ & $1_b$ & $1_c$   \\
                    \hline
                \end{tabular}
                Begin with different elements of same sort value.               \\ \\

                \begin{tabular}{ c | c | c }
                    \hline
                    $1_a$ & $1_b$ & $1_c$  \\ 
                    \hline
                \end{tabular}
                Construct the heap (which does nothing in this case). \\ \\

                \begin{tabular}{ c | c | c }
                    \hline
                    $1_c$ & $1_b$ & \cellcolor{gray}$1_a$  \\ 
                    \hline
                \end{tabular}
                Swap the largest element with the last, and now repeat while ignoring it.   \\ \\

                \begin{tabular}{ c | c | c }
                    \hline
                    $1_c$ & $1_b$ & \cellcolor{gray}$1_a$   \\
                    \hline
                \end{tabular}
                Reconstruct heap (do nothing). \\ \\

                \begin{tabular}{ c | c | c }
                    \hline
                    $1_b$ & \cellcolor{gray}$1_c$ & \cellcolor{gray}$1_a$   \\
                    \hline
                \end{tabular}
                Swap again.                     \\ \\

                \begin{tabular}{ c | c | c }
                    \hline
                    $1_b$ & $1_c$ & $1_a$   \\
                    \hline
                \end{tabular}
                Done. Ordering is not preserved. \\ \\
        \end{enumerate}

    \item
        \begin{lstlisting}
    #include <iostream>
    #include <algorithm>
    #include <ctime>
    #include <cmath>

    using namespace std;

    template <class Item>
    void fix_heap(Item H[], size_t i, size_t n)
    {
        while ((2*i) + 1 <= n )
        {
            size_t largest = i;
            size_t left = 2*i+1;
            size_t right = left + 1;

            largest = (H[i] >= H[left]) ? i : left;
            if (right <= n && H[right] > H[largest])
                largest = right;
            if (largest == i)
                return;
            swap(H[i], H[largest]);
            i = largest;
        }
    }

    template <class Item>
    void make_heap(Item H[], size_t n)
    {
        for (size_t i = ((n-1)/2); i >= 0; --i)
            fix_heap(H, i, n);
    }

    template <class Item>
    void heap_sort(Item H[], size_t n)
    {
        make_heap(H, n-1);
        for (size_t i = n-1; i > 0; --i)
        {
            swap(H[0], H[i]);
            fix_heap(H, 0, i);
        }
    }


    int main()
    {
        srand(time(0));

        size_t n;
        cout << "Enter n: ";
        cin >> n;

        int H[n];

        for (size_t i = 0; i < n; ++i)
            H[i] = i;

        for (size_t i = n-1; i > 0; --i)
        {
            size_t j = (rand() % n);
            swap(H[i], H[j]);
        }

        for (size_t i = 0; i < n; ++i)
            cout << H[i] << " ";
        cout << endl;

        heap_sort(H, n);

        for (size_t i = 0; i < n; ++i)
        {
            cout << H[i] << " ";
            if (H[i] != i)
            {
                cout << "Error!";
                exit(1);
            }
        }
        cout << "Sorted!" << endl;
    }

        \end{lstlisting}            
        
    \item

        \begin{lstlisting}
    //Program to find the most frequently occurring element
    //in an array.
    #include <iostream>
    #include <algorithm>
    #include <vector> 
    #include <ctime>

    void vPrint(std::vector<int> &V){
        for(int i = 0; i < V.size(); i++)
            std::cout << V[i] << ' ';
        std::cout << std::endl;
    }

    void fillVector(std::vector<int> &V, int range, int num){
        for(int i = 0; i < num; i++){
            int n = rand()%range;
            V.push_back(n);
        }
    }

    //Outputs the element that occurs the most and the number of times.
    //If more than one element occur the same number of times, the lower
    //element is chosen.
    int main(){
        srand(time(0));
        std::vector<int> A;
        fillVector(A, 20, 500);
        vPrint(A);
        std::stable_sort(A.begin(), A.end());   //nlgn time
        unsigned maxCount = 0;
        unsigned currentCount = 0;
        unsigned elementIndex = 0;
        unsigned maxIndex = 0;
        for(int i = 0; i < A.size(); i++){  //Linear time
            if(A[i] != A[elementIndex]){ //If we are looking at a new int
                currentCount = 0;
                elementIndex = i;
            }
            currentCount++;
            if(currentCount > maxCount){
                maxCount = currentCount;
                maxIndex = elementIndex;
            }
        }
        std::cout << A[maxIndex] << " occurs " << maxCount 
        << " times." << std::endl;
        return 0;
    }
            \end{lstlisting}
\end{enumerate}
\end{document}
            
