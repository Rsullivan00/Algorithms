\documentclass[11pt]{article}

\usepackage{amsmath}
\usepackage{graphicx}
\usepackage{amssymb}
\usepackage{listings}
\usepackage[margin=1.0in]{geometry}
\usepackage{multicol}

\begin{document}
\noindent   Rick Sullivan   \\
            Professor Tran  \\
            Algorithms-COEN 179 \\
            15 May 2013      \\
                            \\
\centerline{Homework \#4}   \\

\begin{enumerate}
    \item
        Vertex 0: (0,9) \\
        Vertex 1: (1,6) \\
        Vertex 2: (2,5) \\
        Vertex 3: (3,4) \\
        Vertex 4: (7,8) \\
        \\ \\ \\ \\ \\ \\ \\ \\
    \item
        \begin{multicols*}{2}
        TC with Warshall's Algorithm    \\
        k = 0   \\
        \begin{tabular}{c || c | c | c | c | c}
            & 0 & 1 & 2 & 3 & 4    \\ \hline \hline
           0 & 1 & 1 & 0 & 0 & 1    \\ \hline 
           1 & 0 & 1 & 1 & 1 & 0    \\ \hline 
           2 & 0 & 0 & 1 & 1 & 0    \\ \hline 
           3 & 0 & 1 & 0 & 1 & 0    \\ \hline 
           4 & 0 & 0 & 0 & 1 & 1    \\ \hline 
        \end{tabular}
        \\ \\ \\
        k = 1   \\
        \begin{tabular}{c || c | c | c | c | c}
            & 0 & 1 & 2 & 3 & 4    \\ \hline \hline
           0 & 1 & 1 & 1 & 1 & 1    \\ \hline 
           1 & 0 & 1 & 1 & 1 & 0    \\ \hline 
           2 & 0 & 0 & 1 & 1 & 0    \\ \hline 
           3 & 0 & 1 & 1 & 1 & 0    \\ \hline 
           4 & 0 & 0 & 0 & 1 & 1    \\ \hline 
        \end{tabular}
        \\ \\ \\
        k = 2   \\
        \begin{tabular}{c || c | c | c | c | c}
            & 0 & 1 & 2 & 3 & 4    \\ \hline \hline
           0 & 1 & 1 & 1 & 1 & 1    \\ \hline 
           1 & 0 & 1 & 1 & 1 & 0    \\ \hline 
           2 & 0 & 0 & 1 & 1 & 0    \\ \hline 
           3 & 0 & 1 & 1 & 1 & 0    \\ \hline 
           4 & 0 & 0 & 0 & 1 & 1    \\ \hline 
        \end{tabular}
        \vfill
        \columnbreak
        k = 3   \\
        \begin{tabular}{c || c | c | c | c | c}
            & 0 & 1 & 2 & 3 & 4    \\ \hline \hline
           0 & 1 & 1 & 1 & 1 & 1    \\ \hline 
           1 & 0 & 1 & 1 & 1 & 0    \\ \hline 
           2 & 0 & 1 & 1 & 1 & 0    \\ \hline 
           3 & 0 & 1 & 1 & 1 & 0    \\ \hline 
           4 & 0 & 1 & 1 & 1 & 1    \\ \hline 
        \end{tabular}
        \\ \\ \\
        k = 4   \\
        \textbf{Final transitive closure: }  \\
        \begin{tabular}{c || c | c | c | c | c}
            & 0 & 1 & 2 & 3 & 4    \\ \hline \hline
           0 & 1 & 1 & 1 & 1 & 1    \\ \hline 
           1 & 0 & 1 & 1 & 1 & 0    \\ \hline 
           2 & 0 & 1 & 1 & 1 & 0    \\ \hline 
           3 & 0 & 1 & 1 & 1 & 0    \\ \hline 
           4 & 0 & 1 & 1 & 1 & 1    \\ \hline 
        \end{tabular}
        \end{multicols*}

    \item
        Maximum number of topological sorts occurs when we have no restrictions (no edges).
        \\
        A topological sort consists of a list of vertices.  \\
        Therefore we have n = 5 spots, with 5 choices for the first spot, 4 for the second, etc. \\
        So we have 5 * 4 * 3 * 2 * 1 choices = \textbf{120 sort orders}. \\
\end{enumerate}
\end{document}
